\section{Introduction}

\begin{definition*}
	Let $R$ be a commutative ring with unit.
	A (commutative one-dimensional) \emph{formal group law} over $R$ is an element $F\left(x,y\right) \in \taylor{R}{x,y}$, such that:
	\begin{enumerate}
		\item $F\left(x,0\right) = x = F\left(0,x\right)$
		\item $F\left(F\left(x,y\right),z\right)=F\left(x,F\left(y,x\right)\right)$ (associativity)
		\item $F\left(x,y\right)=F\left(y,x\right)$ (commutativity)
	\end{enumerate}
	We denote the \emph{set of formal group laws} over a ring $R$ by $\FGL R$.
\end{definition*}

\begin{definition*}
	Given an homomorphism $\varphi: R \to S$, and $F \in \FGL{R}$
	given by, $F\left(x,y\right) = \sum a_{ij} x^i y^j$,
	we define
	$
		\varphi_*\left(F\right)\left(x,y\right)
		=
		\sum \varphi\left(a_{ij}\right) x^i y^j
	$.
	(This makes $\FGL{\bullet}: \mrm{Ring} \to \mrm{Set}$ into a functor.)
\end{definition*}

\begin{example*}
	The additive formal group law, $F_a\left(x,y\right) = x+y$.
\end{example*}

\begin{example*}
	The multiplicative formal group law, $F_m\left(x,y\right) = x+y+uxy$ for some unit $u\in R$, and specifically $F_m\left(x,y\right) = x+y+xy$.
\end{example*}

\begin{lemma*}
	$p\left(x\right) \in \taylor{R}{x}$ is (multiplicatively) invertible if and only if $p\left(0\right) \in R$ is invertible.
\end{lemma*}
\begin{proof}
	Let $p\left(x\right) = \sum a_n x^n$, and assume $q\left(x\right) = \sum b_n x^n \in \taylor{R}{x}$ is an inverse to $p$, i.e. $pq = 1$. 
	By comparing coefficients it follows that $a_0 b_0 = 1$ (so the first part follows), and $\sum_{k=0}^n a_k b_{n-k} = 0$.
	If $a_0$ is invertible then we can find a suitable $q$, by defining $b_0 = a_0^{-1}$, and $b_n = - a_0^{-1} \left(\sum_{k=1}^n a_k b_{n-k}\right)$ (so the second part follows).
\end{proof}

\begin{lemma*}
	There exists an element $\iota\left(x\right) \in \taylor{R}{x}$ called the \emph{inverse} such that
	$
		F\left(x,\iota\left(x\right)\right)
		= 0
		= F\left(\iota\left(x\right),x\right)
	$.
\end{lemma*}

\begin{definition*}
	An \emph{homomorphism} from $F$ to $G$, two formal group laws over $R$, is a $f \in \taylor{R}{x}$, such that:
	\begin{enumerate}
		\item $f\left(0\right) = 0$
		\item $f\left(F\left(x,y\right)\right) = G\left(f\left(x\right),f\left(y\right)\right)$
	\end{enumerate}
\end{definition*}

\begin{remark*}
	The definition of an homomorphism between formal group laws, turns the collection of formal group laws over a ring into a category,
	Also, given a morphism of rings $\varphi$, the map $\varphi_*$ is actually a functor between the corresponding categories.
\end{remark*}

\begin{lemma*}
	$f: F \to G$ is (compositionally) invertible (i.e. an isomorphism) if and only if $f'\left(0\right)$ is invertible.
\end{lemma*}

\begin{proof}
	It is easy to see the first implication.
	If $f'\left(0\right) = 0$, we can show explicitly that there exists a unique $g$ such that $g\left(f\left(x\right)\right) = x$, and $g'\left(0\right) = \left(f'\left(0\right)\right)^{-1}$. From the very same claim, it follows that there exists an $h$ such that $h\left(g\left(x\right)\right) = x$, it follows that $h\left(x\right) = h\left(g\left(f\left(x\right)\right)\right) = f\left(x\right)$.
\end{proof}

\begin{definition*}
	$f: F \to G$ is a \emph{strict isomorphism} if $f'\left(0\right)=1$.
\end{definition*}

\begin{example*}
	The multiplicative formal group law is strictly isomorphic to the additive formal group law,
	by $f\left(x\right) = u^{-1} \log \left(1+ux\right) = \sum_{n=1}^{\infty} \frac{\left(-u\right)^{n-1} x^n}{n}$:
	\begin{align*}
		f\left(F_m\left(x,y\right)\right)
		&= u^{-1} \log \left(1+u F_m\left(x,y\right)\right)\\
		&= u^{-1} \log \left(1+ux+uy+u^2 xy\right)\\
		&= u^{-1} \log \left(1+ux\right)\left(1+uy\right)\\
		&= u^{-1} \log \left(1+ux\right) + \log \left(1+uy\right)\\
		&= F_a\left(f\left(x\right), f\left(y\right)\right)
	\end{align*}
	(Note that we don't need the $u^{-1}$ to get an isomorphism, but we do need it to get a strict isomorphism.)
\end{example*}

\begin{definition*}
	A strict isomorphism from $F$ to $F_a$ is called a \emph{logarithm}.
\end{definition*}

\begin{lemma*}
	Let $f \in \taylor{R}{x}$ be such that $f\left(0\right) = 0, f'\left(0\right) = 1$ (i.e. $f\left(x\right) = x + \cdots$),
	then there is a unique formal group law $F_f$ over $R$ whose logarithm is $f$.
\end{lemma*}

\begin{proof}
	The condition of being a logarithm means that $f\left(F_f\left(x,y\right)\right) = f\left(x\right) + f\left(y\right)$, or equivalently $F_f\left(x,y\right) = f^{-1}\left(f\left(x\right) + f\left(y\right)\right)$.
	The uniqueness is thus trivial, and being a formal group law is also easy to check.
\end{proof}
