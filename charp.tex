\section{Characteristic $p$}

\begin{remark*}
	The theorem for characteristic 0 is not true over arbitrary rings.
	
	To see this, we define a notion, that will lead us to the concept of height.
	Let $F \in \FGL{R}$. We define $\left[n\right]_F \left(x\right) \in \taylor{R}{x}$, called the \emph{$n$-series} of $F$, recursively:
	$$
		\left[0\right]_F \left(x\right) = 0
		\qquad
		\left[n+1\right]_F \left(x\right) = F\left(x, \left[n\right]_F \left(x\right)\right)
	$$
	Clearly, for $f: F \to G$ we get
	$
		f\left(\left[n\right]_F \left(x\right)\right)
		=
		\left[n\right]_G \left(f\left(x\right)\right)
	$.
	
	For $F_a$ we have $\left[n\right]_{F_a} \left(x\right) = nx$,
	and by induction for $F_m$ we have $\left[n\right]_{F_a} \left(x\right) = \left(1+x\right)^n - 1$.
	Consider them over a field of characteristic $p$, and assume that $f: F_m \to F_a$ is an homomorphism then
	$$
		0
		=
		\left[p\right]_{F_a} \left(f\left(x\right)\right)
		=
		f\left(\left[p\right]_{F_m} \left(x\right)\right)
		=
		f\left(\left(1+x\right)^p-1\right)
		=
		f\left(x^p\right)
	$$
	which means that $f$ is not invertible, thus $F_m$ and $F_a$ are not isomorphic.
\end{remark*}

\begin{lemma*}
	For all $n$, $\left[n\right]_F$ is an endomorphism of $F$.
\end{lemma*}

\begin{proof}
	This amounts to understanding that $\left[n\right]_F\left(x\right)$ is like $nx$.
	It is trivial by definition that $\left[n\right]\left(0\right)=0$.
	The addition by induction. For $n=0$ trivial.
	Now:
	\begin{align*}
		\left[n\right]\left(F\left(x,y\right)\right)
		&= F\left(
			F\left(x,y\right),
			\left[n-1\right]\left(F\left(x,y\right)\right)
		\right)\\
		&= F\left(
			F\left(y,x\right),
			F\left(\left[n-1\right]\left(x\right),\left[n-1\right]\left(y\right)\right)
		\right)\\
		&= F\left(
			y,
			F\left(
				x,
				F\left(\left[n-1\right]\left(x\right),\left[n-1\right]\left(y\right)\right)
			\right)
		\right)\\
		&= F\left(
			y,
			F\left(
				\left[n\right]\left(x\right)
				,\left[n-1\right]\left(y\right)
			\right)
		\right)\\
		&= F\left(
			y,
			F\left(
				\left[n-1\right]\left(y\right),
				\left[n\right]\left(x\right)
			\right)
		\right)\\
		&= F\left(
			\left[n\right]\left(y\right),
			\left[n\right]\left(x\right)
		\right)\\
		&= F\left(
			\left[n\right]\left(x\right),
			\left[n\right]\left(y\right)
		\right)
	\end{align*}
\end{proof}

In what follows in this section, $R$ is an $\mbb{F}_p$-algebra.

\begin{lemma*}
	Let $F,G \in \FGL{R}$, and $f: F \to G$ non-trivial.
	Then $f\left(x\right) = g\left(x^{p^n}\right)$ for some $n$ and $g\in \taylor{R}{x}$ with $g'\left(0\right) \neq 0$, and in particular the leading term of $f$ is $ax^{p^n}$.
\end{lemma*}

\begin{proof}
	If $f'\left(0\right) \neq 0$, we are done.
	Otherwise, we will find a formal group law $\tilde{F}$, and $\tilde{f}: \tilde{F} \to G$, such that $f\left(x\right) = \tilde{f}\left(x^p\right)$. Since $f$ is non-trivial, and the least non-zero degree is lowered by this process, this process must terminate after a finite amount of stages.
	So suppose $f'\left(0\right) = 0$.
	
	First we claim that $f'\left(x\right) = 0$.
	Deriving $f\left(F\left(x,y\right)\right) = G\left(f\left(x\right),f\left(y\right)\right)$ by $y$ and setting $y=0$, we get
	$
		f'\left(F\left(x,0\right)\right) F_2\left(x,0\right)
		= G_2\left(f\left(x\right),f\left(0\right)\right) f'\left(0\right)
	$
	remembering that $F\left(x,0\right) = x, F_2\left(x,0\right) = 1, f'\left(0\right) = 0$,
	we conclude that $f'\left(x\right) = 0$.
	Now, write $f\left(x\right) = \sum a_n x^n$,
	from $f'\left(x\right) = 0$ it follows that $na_n = 0$ for all $n$, thus $a_n = 0$ for all $p \nmid n$.
	So we can define $\tilde{f}$, by $f\left(x\right) = \tilde{f}\left(x^p\right)$.
	
	Denote by $\varphi: R \to R$ the Frobenius endomorphism $\varphi\left(x\right) = x^p$.
	Define $\tilde{F} = \varphi_*\left(F\right)$.
	It follows that
	$$
		\tilde{f}\left(\tilde{F}\left(x^p, y^p\right)\right)
		= \tilde{f}\left(F\left(x, y\right)^p\right)
		= f\left(F\left(x, y\right)\right)
		= G\left(f\left(x\right),f\left(y\right)\right)
		= G\left(\tilde{f}\left(x^p\right),\tilde{f}\left(y^p\right)\right)
	$$
	thus
	$
		\tilde{f}\left(\tilde{F}\left(x, y\right)\right)
		= G\left(\tilde{f}\left(x\right),\tilde{f}\left(y\right)\right)
	$
	(since these are just formal power series, so just rename the variables),
	and it follows that $\tilde{f}: \tilde{F} \to G$ is the desired homomorphism.
\end{proof}

\begin{definition*}
	The \emph{height} of $F \in \FGL{R}$ is defined as follows:
	if $\left[p\right]_F\left(x\right) = 0$, the height is $\infty$,
	otherwise it is the unique $n \in \mbb{N}$ such that $\left[p\right]_F\left(x\right) = g\left(x^{p^n}\right)$ with $g'\left(0\right) \neq 0$.
\end{definition*}

\begin{lemma*}
	The height is an isomorphism invariant.
\end{lemma*}

\begin{proof}
	Let $f: F \to G$ be an isomorphism.
	We've seen that in that case
	$
		f\left(\left[n\right]_F \left(x\right)\right)
		=
		\left[n\right]_G \left(f\left(x\right)\right)
	$.
	Since $f$ is an isomorphism, $f'\left(0\right)$ is a unit, the least non-zero degree is conserved and the result follows.
\end{proof}

\begin{theorem*}
	For each $1 \leq n \leq \infty$ there exists a formal group law $F_n$ of height $n$.
\end{theorem*}

\begin{theorem*}
	Over an algebraically closed field, there is a unique formal group law of each height $1 \leq n \leq \infty$.
\end{theorem*}
