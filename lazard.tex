\section{The Lazard Ring}

\begin{definition*}
	Given an homomorphism $\varphi: R \to S$,
	and a formal group law over $R$, $F\left(x,y\right) = \sum a_{ij} x^i y^j$,
	we define the \emph{base change} by
	$
		\varphi^*\left(F\right)\left(x,y\right)
		=
		\sum \varphi\left(a_{ij}\right) x^i y^j
	$.
	In fact, that is a functor, defined similarly for morphisms.
\end{definition*}

\begin{theorem*}
	There is a ring $L$, called the \emph{Lazard ring}, and a formal group law over it $F_\mrm{univ}$, called \emph{the universal formal group law},
	such that for every ring $R$ the map
	$$
		\hom_\mrm{Ring}\left(L,R\right)
		\to
		\FGL R
		\qquad
		\varphi \mapsto \varphi^*\left(F_\mrm{univ}\right)
	$$
	is one-to-one and onto.
	That is, the functor $\mrm{FGL}: \mrm{Ring} \to \mrm{Set}$ is corepresentable by $L$.
\end{theorem*}

\begin{proof}
	Look at the ring $\tilde{L} = \mbb{Z}\left[c_{ij}\right]$,
	and
	$
		\tilde{F}_\mrm{univ}\left(x,y\right)
		=
		\sum c_{ij} x^i y^j
		\in
		\taylor{\tilde{L}}{x,y}
	$.
	There are various relations obtained from the definition of a formal group law, e.g. $c_{0j} = 0 = c_{i0}$.
	Denote by $I$ the ideal generated by these relations, and define $L=\tilde{L}/I$,
	and
	$
		F_\mrm{univ}\left(x,y\right)
		=
		\sum \left(c_{ij} + I\right) x^i y^j
		\in
		\taylor{L}{x,y}
	$,
	which satisfies the definition of a formal group law over $L$ by construction.
	The map being one-to-one is trivial.
	Given a formal group law $F\left(x,y\right) = \sum a_{ij} x^i y^j$,
	we can define $\tilde{\varphi}: \tilde{L} \to R$ by $\tilde{\varphi}\left(c_{ij}\right) = a_{ij}$,
	It is clear that if factors through $L$, to a map $\varphi: L \to R$, and that the base w.r.t $\varphi^*\left(F_\mrm{univ}\right) = F$,
	therefore it is onto.
\end{proof}

\todo{Lazard's theorem and grading of $L$}