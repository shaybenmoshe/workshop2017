\ncmd{\LR}[1]{\mbb{Z}\left[#1_1, #1_2, \dotsc\right]}
\ncmd{\LRQ}[1]{\mbb{Q}\left[#1_1, #1_2, \dotsc\right]}

\section{The Lazard Ring}

\begin{theorem*}
	There is a ring $L$, called the \emph{Lazard ring}, and a formal group law over it $F_\mrm{univ}$, called \emph{the universal formal group law},
	such that for every ring $R$ the map
	$$
		\hom_\mrm{Ring}\left(L,R\right)
		\to
		\FGL R
		\qquad
		\varphi \mapsto \varphi_*\left(F_\mrm{univ}\right)
	$$
	is one-to-one and onto.
	That is, the functor $\mrm{FGL}: \mrm{Ring} \to \mrm{Set}$ is corepresentable by $L$.
\end{theorem*}

\begin{proof}
	Look at the ring $\tilde{L} = \mbb{Z}\left[c_{ij}\right]$,
	and
	$
		\tilde{F}_\mrm{univ}\left(x,y\right)
		=
		\sum c_{ij} x^i y^j
		\in
		\taylor{\tilde{L}}{x,y}
	$.
	There are various relations obtained from the definition of a formal group law, e.g. $c_{0j} = 0 = c_{i0}$.
	Denote by $I$ the ideal generated by these relations, and define $L=\tilde{L}/I$,
	and
	$
		F_\mrm{univ}\left(x,y\right)
		=
		\sum \left(c_{ij} + I\right) x^i y^j
		\in
		\taylor{L}{x,y}
	$,
	which satisfies the definition of a formal group law over $L$ by construction.
	The map being one-to-one is trivial.
	Given a formal group law $F\left(x,y\right) = \sum a_{ij} x^i y^j$,
	we can define $\tilde{\varphi}: \tilde{L} \to R$ by $\tilde{\varphi}\left(c_{ij}\right) = a_{ij}$.
	It is clear that $\tilde{\varphi}$ is $0$ on $I$ (since the coefficients $F$ satisfy the relations), so that it factors to a map $\varphi: L \to R$, and clearly $\varphi_*\left(F_\mrm{univ}\right) = F$,
	therefore it is onto.
\end{proof}

We can define grading on $L$, by first define a grading on $\tilde{L}$.
Assume that $\left|x\right|,\left|y\right| = d$, and require that $\left|F_\mrm{univ}\left(x,y\right)\right| = d$, then $d = \deg\left(c_{ij}\right) + di +dj$.
It is convenient (specifically for algebraic topology) to choose $d=2$, thus $\left|c_{ij}\right| = 2\left(i+j-1\right)$.
It is also true that all relations in the definition of a formal group law compare values of the same degree, thus the grading descends to $L$.
(Also note that $c_{00}=0$ so it is non-negatively graded.)

\begin{theorem*}[Lazard]
	$L \cong \LR{t}$ where $\left|t_i\right| = 2i$.
\end{theorem*}

Look at the ring $\LR{b}$ where $\left|b_i\right| = 2i$,
and define $g\left(x\right) = x + b_1 x^2 + b_2 x^3 + \dotsc$.
We showed before that $g^{-1}\left(g\left(x\right) + g\left(y\right)\right)$ defines a formal group,
and it is corepresented by a map $\phi: L \to \LR{b}$ (compatible with the grading).

Let $I,J$ be the ideals consisting of elements of positive degree in $L,\LR{b}$ respectively.
It is clear that $J/J^2$ is a free abelian group with generators $b_i$.

\begin{lemma*}
	$\phi$ induces an injection
	$\left(I/I^2\right)_{2n} \to \left(J/J^2\right)_{2n} \cong \mbb{Z}$,
	and the image is $p\mbb{Z}$ if $n+1=p^f$, and $\mbb{Z}$ otherwise.
\end{lemma*}

In particular it follows that $\left(I/I^2\right)_{2n} \cong \mbb{Z}$,
and we can choose generators, and lift them to homogeneous $t_n \in I_{2n} = L_{2n}$.
This naturally defines a map $\theta: \LR{t} \to L$.

\begin{lemma*}
	$\theta$ is surjective.
\end{lemma*}

\begin{proof}
	Easy induction on the degrees.
	Note that we have relation $c_{01}, c_{10} = 1$, so the base case follows.
	Elements of $\left(I^2\right)_{2n}$ are generated by products of elements of degrees $1 \leq d < 2n$, which are in $\im \theta$ by induction, thus $\left(I^2\right)_{2n} \subset \im \theta$.
	Since $t_i \in \im \theta$ is a generator of $\left(I/I^2\right)_{2n}$, it follows that $L_{2n} = I_{2n} \subset \im \theta$.
\end{proof}

\begin{lemma*}
	$\psi = \phi\theta: \LR{t} \to L \to \LR{b}$ is injective, and in particular $\theta$ is injective.
\end{lemma*}

\begin{proof}
	Since they are torsion-free, it is sufficient to prove that
	$\psi_\mbb{Q} = \phi_\mbb{Q}\theta_\mbb{Q}: \LRQ{t} \to L\otimes\mbb{Q} \to \LRQ{b}$
	is an isomorphism.
	We define $J'$ as the ideal of elements of positive degree in $\LRQ{t}$.
	Then $J'/\left(J'\right)^2$ is a $\mbb{Q}$ vector space with basis $t_i$.
	$\theta_\mbb{Q}$ is clearly surjective since $\theta$ is.
	By the result on the image of $\phi$ it is also clear that $\phi_\mbb{Q}$ is surjective.
	Thus the composition $\psi_\mbb{Q}$ is surjective.
	Since the rings in the domain and the range are finite dimensional $\mbb{Q}$ with surjective $\mbb{Q}$-linear map between them, it follows that it is an isomorphism level-wise, and thus globally.
\end{proof}

\begin{proof}[Proof of Lazard's theorem]
	The map $\theta: \LR{t} \to L$ was shown to be injective and surjective.
\end{proof}
